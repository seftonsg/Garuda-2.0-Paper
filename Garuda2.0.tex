%% 
%% This is file `sample-sigchi.tex',
%% generated with the docstrip utility.
%%
%% The original source files were:
%%
%% samples.dtx  (with options: `sigchi')
%% 
%% IMPORTANT NOTICE:
%% 
%% For the copyright see the source file.
%% 
%% Any modified versions of this file must be renamed
%% with new filenames distinct from sample-sigchi.tex.
%% 
%% For distribution of the original source see the terms
%% for copying and modification in the file samples.dtx.
%% 
%% This generated file may be distributed as long as the
%% original source files, as listed above, are part of the
%% same distribution. (The sources need not necessarily be
%% in the same archive or directory.)
%%
%% The first command in your LaTeX source must be the \documentclass command.
% \documentclass[sigchi]{acmart}
\documentclass[sigconf,usenames,dvipsnames,svgnames,table]{acmart}

%% Standard Preamble
    \usepackage{xspace} % Spacing for sysname

%%
%% \BibTeX command to typeset BibTeX logo in the docs
\AtBeginDocument{%
  \providecommand\BibTeX{{%
    \normalfont B\kern-0.5em{\scshape i\kern-0.25em b}\kern-0.8em\TeX}}}

%% Rights management information.  This information is sent to you
%% when you complete the rights form.  These commands have SAMPLE
%% values in them; it is your responsibility as an author to replace
%% the commands and values with those provided to you when you
%% complete the rights form.
\acmConference[CASES]{ESWEEK}{???? 2020}{????}
\setcopyright{acmcopyright}
\copyrightyear{2020}
\acmYear{2020}
% \acmDOI{10.1145/1122445.1122456}

%%
%% Submission ID.
%% Use this when submitting an article to a sponsored event. You'll
%% receive a unique submission ID from the organizers
%% of the event, and this ID should be used as the parameter to this command.
%%\acmSubmissionID{123-A56-BU3}

%%
%% The majority of ACM publications use numbered citations and
%% references.  The command \citestyle{authoryear} switches to the
%% "author year" style.
%%
%% If you are preparing content for an event
%% sponsored by ACM SIGGRAPH, you must use the "author year" style of
%% citations and references.
%% Uncommenting
%% the next command will enable that style.
%%\citestyle{acmauthoryear}

%%
%% end of the preamble, start of the body of the document source.
\begin{document}

%%
%% Custom definitions
\def \sysname {\textsc{G2}\xspace}
\def \oldname {\textsc{GARUDA}\xspace}
\def \defeq   {\triangleq}
\newcommand{\interp}[1]{\llbracket #1 \rrbracket}


%% These commands are for a PROCEEDINGS abstract or paper.
% \acmConference[Woodstock '18]{Woodstock '18: ACM Symposium on Neural
%   Gaze Detection}{June 03--05, 2018}{Woodstock, NY}
% \acmBooktitle{Woodstock '18: ACM Symposium on Neural Gaze Detection,
%   June 03--05, 2018, Woodstock, NY}
% \acmPrice{15.00}
% \acmISBN{978-1-4503-XXXX-X/18/06}

%%
%% The "title" command has an optional parameter,
%% allowing the author to define a "short title" to be used in page headers.
\title{\sysname}

%%
%% The "author" command and its associated commands are used to define
%% the authors and their affiliations.
%% Of note is the shared affiliation of the first two authors, and the
%% "authornote" and "authornotemark" commands
%% used to denote shared contribution to the research.

%%Check name legality.
\author{Sage Sefton}
\email{ss557415@ohio.edu}
\affiliation{%
  \institution{Ohio University}
}

\author{Avinash Karanth}
  \email{karanth@ohio.edu}
\affiliation{%
  \institution{Ohio University}
%   \city{Athens}
%   \state{Ohio}
%   \postcode{45701}
}

\author{Gordon Stewart}
  \email{gstewart@ohio.edu}
\affiliation{%
  \institution{Ohio University}
%   \city{Athens}
%   \state{Ohio}
%   \postcode{45701}
}


%%
%% By default, the full list of authors will be used in the page
%% headers. Often, this list is too long, and will overlap
%% other information printed in the page headers. This command allows
%% the author to define a more concise list
%% of authors' names for this purpose.
\renewcommand{\shortauthors}{Sefton and Karanth et. al.}

\begin{abstract}

Runtime monitors to enforce processor security policies are a widely researched field.
Many systems tackle particular issues such as non-interference, cache side-channel attacks, or speculation exploits \cite{2017secverilogbl, 2019stt}.
As the attack model is ever-mutating and often undetected, these solutions can only ever solve part of the problem.
Some solutions propose generic monitors based on programmable hardware monitors \cite{2014pump} or languages for describing arbitrary policies \cite{2018garuda, 2014netkat}.
We build on the idea of verifiable policy design and empower \oldname to support Speculative and Out-of-Order Processors.
The benefits of Speculative and OOP microarchitectures are one of the most significant advancements in recent latency-mitigation research.
\sysname benefits from a refined information model and taint tracking, implemented without changes to the language itself.
These additions incur marginal processor overhead, and ring true to the spatial and power efficiency goals of the past.
We inject monitors into only one part of the processor and automate the comparator for a more streamlined design.
Finally, we will release this code, compiler, and language on GitHub.

\end{abstract}

%%
%% The code below is generated by the tool at http://dl.acm.org/ccs.cfm.
%% Please copy and paste the code instead of the example below.
% %%
% \begin{CCSXML}
% <ccs2012>
%  <concept>
%   <concept_id>10010520.10010553.10010562</concept_id>
%   <concept_desc>Computer systems organization~Embedded systems</concept_desc>
%   <concept_significance>500</concept_significance>
%  </concept>
%  <concept>
%   <concept_id>10010520.10010575.10010755</concept_id>
%   <concept_desc>Computer systems organization~Redundancy</concept_desc>
%   <concept_significance>300</concept_significance>
%  </concept>
%  <concept>
%   <concept_id>10010520.10010553.10010554</concept_id>
%   <concept_desc>Computer systems organization~Robotics</concept_desc>
%   <concept_significance>100</concept_significance>
%  </concept>
%  <concept>
%   <concept_id>10003033.10003083.10003095</concept_id>
%   <concept_desc>Networks~Network reliability</concept_desc>
%   <concept_significance>100</concept_significance>
%  </concept>
% </ccs2012>
% \end{CCSXML}

% \ccsdesc[500]{Computer systems organization~Embedded systems}
% \ccsdesc[300]{Computer systems organization~Redundancy}
% \ccsdesc{Computer systems organization~Robotics}
% \ccsdesc[100]{Networks~Network reliability}

%%
%% Keywords
\keywords{hardware security, software verification}

%%
%% This command processes the author and affiliation and title
%% information and builds the first part of the formatted document.
\maketitle

\section{Introduction}
\sysname Intro
\cite{2011gem5sim}
\cite{2019stt}

\section {Syntax}
        
        % Definitions table
        \begin{figure}
          \centering
          \begin{tabular}{l r c l}
            Instruction Fields   & $f_{i}$   & $::=$ & $f_{i_{1}} \mid \dots \mid f_{i_{k}}$\\
            Result Fields   & $f_{r}$   & $::=$ & $f_{r_{1}} \mid \dots \mid f_{r_{k}}$\\
            Fields          & $f$       & $::=$ & $f_i \mid f_r$ \\
            Instructions    & $i$       & $::=$ & $\{f_{i_{1}} = v_{i_{1}} ,\ \dots\ ,\ f_{i_{k}} = v_{i_{k}}\}$\\
            Results         & $r$       & $::=$ & $\{f_{r_{1}} = v_{r_{1}} ,\ \dots\ ,\ f_{r_{k}} = v_{r_{k}}\}$
          \end{tabular}
          \caption{Definitions in \oldname.}
          \label{fig:garuda:defn}
        \end{figure}
        
        % Predicates table
        \begin{figure}
          \centering
          \begin{tabular}{l c r l l}
            Predicates  & $a,b$ & $::=$  & $0$         & \textit{False}    \\
                        &       & $\mid$ & $1$         & \textit{Identity} \\
                        &       & $\mid$ & $f=n$       & \textit{Test}     \\  
                        &       & $\mid$ & $a + b$     & \textit{Sum}      \\
                        &       & $\mid$ & $a \cdot b$ & \textit{Product}  \\
                        &       & $\mid$ & $\neg \ a$  & \textit{Negation}
          \end{tabular}
          \caption{Syntax of predicates in \oldname.}
          \label{fig:garuda:synt:pred}
        \end{figure}
        
        % Policies table
        \begin{figure}
          \centering
          \begin{tabular}{l c r l l}
            Policies    & $p,q$ & $::=$  & $\mathbf{test}(a)$ & \textit{Test}     \\
                        &       & $\mid$ & $act(p)$           & \textit{Slice Actions}    \\
                        &       & $\mid$ & $res(p)$           & \textit{Slice Results}    \\
                        &       & $\mid$ & $inj_{i}$          & \textit{Injection Action} \\
                        &       & $\mid$ & $inj_{r}$          & \textit{Injection Result} \\
                        &       & $\mid$ & $f \leftarrow n$   & \textit{Update}   \\
                        &       & $\mid$ & $p + q$            & \textit{Choice}   \\
                        &       & $\mid$ & $p \cdot q$        & \textit{Sequential Concatenation}
          \end{tabular}
          \caption{Syntax of policies in \oldname.}
          \label{fig:garuda:synt:pol}
        \end{figure}
        
\section {Semantics}
% Semantics of Predicates
        \begin{figure}
          \centering
          \begin{align*}
            % Theory
            % TODO: can this be shaded or highlighted in some way?
            \interp{ \cdot }\ 
              :\ \ &
              \mathbf{Stream}(I)\times \mathbf{Stream}(R) \rightarrow \\
              & P(\mathbf{Stream}(I))\times P(\mathbf{Stream}(R)) 
              \\
              \\
            % Falsity
            \interp{ 0 }(-, -)
              \defeq\ &
              (\emptyset , \emptyset)
              \\ %or null-stream?
            % Identity
            \interp{ 1 }(\mathit{is}, \mathit{rs})
              \defeq\ &
              (\{\mathit{is}\},\{\mathit{rs}\})
              \\
            % Test
            \interp{ f=n }(\mathit{is}, \mathit{rs})
              \defeq\ &
              (\mathsf{filter}\ (f=n)\ \{\mathit{is}\},\
               \mathsf{filter}\ (f=n)\ \{\mathit{rs}\}) 
              \\
            % Sum
            \interp{ a + b }(\mathit{is}, \mathit{rs})
              \defeq\ &
              \interp { a }(\mathit{is}, \mathit{rs})\cup
              \interp { b }(\mathit{is}, \mathit{rs}) \\
              &\mathsf{where}\ (S_i^1, S_r^1)\cup (S_i^2, S_r^2)\defeq
                (S_i^1\cup S_i^2, S_r^1\cup S_r^2)\\
            % Product
            \interp { a \cdot b }(\mathit{is}, \mathit{rs})
              \defeq\ &
              \interp { a }(\mathit{is}, \mathit{rs})\cap
              \interp { b }(\mathit{is}, \mathit{rs}) \\
              &\mathsf{where}\ (S_i^1, S_r^1)\cap (S_i^2, S_r^2)\defeq
                (S_i^1\cap S_i^2, S_r^1\cap S_r^2)\\
            % Negation
            \interp { \neg a }(\mathit{is}, \mathit{rs})
              \defeq\ &
              \mathsf{let}\ (S_i, S_r) = \interp {a}(\mathit{is}, \mathit{rs}) \\
              &\mathsf{in}\ (\{\mathit{is}\} - S_i, \{\mathit{rs}\} - S_r)
          \end{align*}
          \caption{The semantics of predicates in \oldname.}
          \label{fig:garuda:sem:pred}
        \end{figure}
        % Semantics of Policies
        \begin{figure}
          \centering
          \begin{align*}
            % Theory
            % TODO: can this be shaded or highlighted in some way?
            \interp{ \cdot }\ 
              :\ \ &
              \mathbf{Stream}(I)\times \mathbf{Stream}(R) \rightarrow \\
              & P(\mathbf{Stream}(I))\times P(\mathbf{Stream}(R)) 
              \\
              \\
            % Action-only Slice
            \interp { act(p) }(\mathit{is}, \mathit{rs})
              \defeq\ &
              \mathsf{let}\ (S_i,S_r) = \interp {p}(\mathit{is}, \mathit{rs})\ 
              \mathsf{in}\ (S_i, \{\mathit{rs}\})
              \\
            % Result-only Slice
            \interp { res(p) }(\mathit{is}, \mathit{rs})
              \defeq\ &
              \mathsf{let}\ (S_i,S_r) = \interp {p}(\mathit{is}, \mathit{rs})\ 
              \mathsf{in}\ (\{\mathit{is}\}, S_r)
              \\
            % Injection Action
            \interp { inj_{i}(i) }(\mathit{is}, \mathit{rs})
              \defeq\ &
              (\{\mathit{i : is}\}, \{\mathit{rs}\}) 
              %or ::?  ; is sequential
              \\
            % Injection Result
            \interp { inj_{r}(r) }(\mathit{is}, \mathit{rs})
              \defeq\ &
              (\{\mathit{is}\},\{ \mathit{r : rs}\})
              \\
            % Modification
            \interp { f \leftarrow n }(is, rs)
              \defeq\ &
              (\mathsf{map}\ (f\leftarrow n)\ \{is\},\
               \mathsf{map}\ (f\leftarrow n)\ \{rs\})
              \\ % could use editing?
            % Policy Union
            \interp { p + q }(\mathit{is}, \mathit{rs})
              \defeq\ &
              \interp { p }(\mathit{is}, \mathit{rs})\cup
              \interp { q }(\mathit{is}, \mathit{rs}) \\
              &\mathsf{where}\ (S_i^1, S_r^1)\cup (S_i^2, S_r^2)\defeq
                (S_i^1\cup S_i^2, S_r^1\cup S_r^2)\\
              %A union of the streams gives a nondeterministic machine.
            % Policy Concatnation
            \interp { p \cdot q }(\mathit{is}, \mathit{rs})
              \defeq\ &
              \mathsf{let}\ (S_i, S_r) = \interp{p}(is, rs)\\
              &\mathsf{in}\ \bigcup \{\interp{q}(\mathit{is}',\mathit{rs}')\ |\ \mathit{is}'\in S_i, \mathit{rs'}\in S_r\}\\
              \\
            %Defn. of Filter
            \interp{\mathsf{filter}\ f}(S)
              \defeq\ & \{l \in S\ |\ f(l) = \mathsf{true}\}\\
            %Defn. of Map
            \interp{\mathsf{map}\ g}(S)
              \defeq\ &
              \{ g(l)\ |\ l\in S \}
          \end{align*}
          \caption{The semantics of policies in \oldname.}
          \label{fig:garuda:sem:pol}
        \end{figure}

%% Omitted: Make sure sections use standard sectioning conventions (\section, \subsepction, \paragraph...)

%%
%% The acknowledgments section is defined using the "acks" environment
%% (and NOT an unnumbered section). This ensures the proper
%% identification of the section in the article metadata, and the
%% consistent spelling of the heading.
\begin{acks}
To ...
\end{acks}

%%
%% The next two lines define the bibliography style to be used, and
%% the bibliography file.
\bibliographystyle{ACM-Reference-Format}
\bibliography{Garuda2.0}

%%
%% If your work has an appendix, this is the place to put it.
\appendix

\section{Research Methods}

\subsection{Part One}

Appendix A1

\subsection{Part Two}

Appendix A2

\section{Online Resources}

Appendix B

\end{document}
\endinput
