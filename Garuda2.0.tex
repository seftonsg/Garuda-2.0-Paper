%% 
%% This is file `sample-sigchi.tex',
%% generated with the docstrip utility.
%%
%% The original source files were:
%%
%% samples.dtx  (with options: `sigchi')
%% 
%% IMPORTANT NOTICE:
%% 
%% For the copyright see the source file.
%% 
%% Any modified versions of this file must be renamed
%% with new filenames distinct from sample-sigchi.tex.
%% 
%% For distribution of the original source see the terms
%% for copying and modification in the file samples.dtx.
%% 
%% This generated file may be distributed as long as the
%% original source files, as listed above, are part of the
%% same distribution. (The sources need not necessarily be
%% in the same archive or directory.)
%%%%%%%%%%%%%%%%%%%%%%%%%%%%%%%%%%%%%%%%%%%%%%%%%%%%%%%%%%%%

%%%%%%%%%%%%%%%%%%%%%%
%%%%%% PREAMBLE %%%%%%
%%%%%%%%%%%%%%%%%%%%%%
\documentclass[sigconf,usenames,dvipsnames,svgnames,table]{acmart}

%% Packages %
  % Standard Packages %
    \usepackage{pdfpages} % Including PDFs. See Graphicx for more info
    \usepackage{amsmath}  % \mvert and other relation symbol
    \usepackage{amssymb}  % Semantic delta=
    \usepackage{amsthm}   % \begin{proof} and related sections
    \usepackage{xspace}   % Spacing for sysname
    \usepackage{dsfont}   % Fancy set letters
    \usepackage{xcolor}   % Things that need attention, and listingsconfig
  % Coding Packages %
    \usepackage{listings}
% \usepackage{sectsty}

\lstdefinelanguage{myC}{
  language=C,
  basicstyle=\ttfamily\singlespacing,
  showspaces=false,              
  showstringspaces=false,        
  showtabs=false,    
  tabsize=2,                      % sets default tabsize to 2 spaces
  captionpos=b,                   % sets the caption-position to bottom
  breaklines=true,                % sets automatic line breaking
  breakatwhitespace=false,
  escapeinside={\%*}{*)},        
  keywordstyle=\bfseries\color{black},    % keyword style
  numberstyle=\tiny\color{gray},
%  numbers=left,
%  frame=single,                  % adds a frame around the code
%  rulecolor=\color{black},
%  stepnumber=1,                 
%  numbersep=5pt,                
%  backgroundcolor=\color{white}, 
%  commentstyle=\color{dkgreen},  % comment style
%  stringstyle=\color{mauve},     % string literal style
}
% A version of the above myC, but without singlespacing
\lstdefinelanguage{myinlineC}{
  language=myC,
  basicstyle=\ttfamily
}
\renewcommand{\ttdefault}{pcr}    % enables bold monospaced font
\lstdefinelanguage[x86gasm]{Assembler}[x86masm]{Assembler}{%
,basicstyle=\ttfamily\singlespacing
,morekeywords={rax,rbx,rcx,rdx,rip,rdi,rsi,rsp,subq,decl,movq
              ,movl,xorl,imull,popq,popl,pushl}%
,morekeywords=[2]{.file,.section,.string,.text,.globl,.cfi_startproc
                 ,.cfi_def_cfa_offset,.cfi_endproc,.size,.ident}%
}
%% \definecolor{cmmtcolor}{named}{DarkGreen}
\definecolor{cmmtcolor}{named}{OliveGreen}
\lstdefinelanguage{Coq}{
,morekeywords={match,end,Definition,Inductive,Lemma,Theorem,Record,
               Hypothesis,Variable,Section,End,case,of,if,then,else,is,let,in,do,return,with,Extract,Constant,Inlined,Inline,Extraction,Fixpoint,Program,Function,Fix,Class,Local,Output,Input}
,morecomment=[s]{(*}{*)}
,keywordstyle=\bfseries\color{MidnightBlue}
,commentstyle={\color{cmmtcolor}}
,basicstyle=\sffamily
,columns=fullflexible
,numberstyle=\tiny\color{gray}
,escapeinside={@}{@}
,literate=
    {:=}{{$\defeq\;$}}1
    {<-}{{$\leftarrow\;$}}1
    {=>}{{$\Rightarrow\;$}}1
    {->}{{$\rightarrow\;$}}1
    {<->}{{$\leftrightarrow\;$}}1
    {<==}{{$\leq\;$}}1
    {\\/}{{$\vee\;$}}1
    {/\\}{{$\land\;$}}1
    {ffun}{{$\mathsf{ffun}$}}1    
    {fun}{{$\lambda$}}1
    {forall}{{$\forall$}}1
    {exists}{{$\exists$}}1
    {Z}{{$\mathbb{Z}$}}1
    {Z0}{{$\mathbb{Z}_0$}}1
    {<=}{{$\leq\;$}}1
    {>=}{{$\geq\;$}}1
    {<>}{{$\neq\;$}}1                
}

%% Verilog
\definecolor{vgreen}{RGB}{104,180,104}
\definecolor{vblue}{RGB}{49,49,255}
\definecolor{vorange}{RGB}{255,143,102}

\lstdefinestyle{verilog-style}
{
    language=Verilog,
    basicstyle=\small\sffamily,
    keywordstyle=\bfseries\color{MidnightBlue},
    identifierstyle=\color{black},
    commentstyle=\color{cmmtcolor},
    tabsize=8,
    moredelim=*[s][\colorIndex]{[}{]},
    literate=*{:}{:}1
}

\makeatletter
\newcommand*\@lbracket{[}
\newcommand*\@rbracket{]}
\newcommand*\@colon{:}
\newcommand*\colorIndex{%
   \edef\@temp{\the\lst@token}%
        \ifx\@temp\@lbracket \color{black}%
            \else\ifx\@temp\@rbracket \color{black}%
                \else\ifx\@temp\@colon \color{black}%
                    \else \color{vorange}%
                        \fi\fi\fi
                        }
\makeatother
                        

    \lstset{language=Coq}
    \usepackage[framemethod=TikZ]{mdframed}
    \mdfsetup{frametitlealignment=\center}
  % Figure Packages %
    \usepackage{caption}  % Captions
    \usepackage{framed}   % Shaded figures
    \usepackage{epsfig}   % Import PNG figure
    \usepackage{stmaryrd} % Semantics [[ and ]]
  % Specific Thesis Packages %
    \usepackage{contour}
    \usepackage{geometry}
    \usepackage{indentfirst} % Indent after declaring a new (sub)section
    \usepackage[toc,page]
               {appendix} % Add appendix to ToC
    % \usepackage{fontenc}
    % \usepackage{setspace} % Set line spacing
    % \usepackage{soul}     % Set highlight colors
    % \usepackage{tocloft}  % Set up Tables of other things
    % \usepackage{fontspec} % Specify Font
    % \usepackage{ragged2e} % Justification
  % Extra %
    % \usepackage{enumitem}
    % \usepackage{tabularx}
    % \usepackage{graphicx}
    % \usepackage{proof}  % To write Proofs (included with amsthm?)

% Custom Definitions %
  \newcommand{\interp}[1]{\llbracket #1 \rrbracket}
  \newcommand{\obf}[1]{#1^\mathbf{o}}
  \def \sysname {\textsc{G2}\xspace}
  \def \oldname {\textsc{GARUDA}\xspace}
  \def \defeq   {\triangleq}
  \def \bksl    {\textbackslash}
  \def \latex   {\LaTeX\xspace}

%%
%% \BibTeX command to typeset BibTeX logo in the docs
\AtBeginDocument{%
  \providecommand\BibTeX{{%
    \normalfont B\kern-0.5em{\scshape i\kern-0.25em b}\kern-0.8em\TeX}}}

%% Rights management information.  This information is sent to you
%% when you complete the rights form.  These commands have SAMPLE
%% values in them; it is your responsibility as an author to replace
%% the commands and values with those provided to you when you
%% complete the rights form.
\acmConference[TODO????]{JETC Special Issue on Emerging Challenges and Solutions in Hardware Security}{15 June 2020}{????}
\setcopyright{acmcopyright}
\copyrightyear{2020}
\acmYear{2020}
% \acmDOI{10.1145/1122445.1122456}

%%
%% Submission ID.
%% Use this when submitting an article to a sponsored event. You'll
%% receive a unique submission ID from the organizers
%% of the event, and this ID should be used as the parameter to this command.
%%\acmSubmissionID{123-A56-BU3}

%%
%% The majority of ACM publications use numbered citations and
%% references.  The command \citestyle{authoryear} switches to the
%% "author year" style.
%% \citestyle{acmauthoryear}

%%
%% end of the preamble, start of the body of the document source.
\begin{document}



%% These commands are for a PROCEEDINGS abstract or paper.
% \acmConference[Woodstock '18]{Woodstock '18: ACM Symposium on Neural
%   Gaze Detection}{June 03--05, 2018}{Woodstock, NY}
% \acmBooktitle{Woodstock '18: ACM Symposium on Neural Gaze Detection,
%   June 03--05, 2018, Woodstock, NY}
% \acmPrice{15.00}
% \acmISBN{978-1-4503-XXXX-X/18/06}

%%
%% The "title" command has an optional parameter,
%% allowing the author to define a "short title" to be used in page headers.
\title{\sysname}

%%
%% The "author" command and its associated commands are used to define
%% the authors and their affiliations.

%%Check name legality.
\author{Sage Sefton}
\email{ss557415@ohio.edu}
\affiliation{%
  \institution{Ohio University}
}

\author{Avinash Karanth}
  \email{karanth@ohio.edu}
\affiliation{
  \institution{Ohio University}
%   \city{Athens}
%   \state{Ohio}
%   \postcode{45701}
}


%%
%% By default, the full list of authors will be used in the page
%% headers. Often, this list is too long, and will overlap
%% other information printed in the page headers. This command allows
%% the author to define a more concise list
%% of authors' names for this purpose.
\renewcommand{\shortauthors}{Sefton and Karanth}

\begin{abstract}

Runtime monitors to enforce processor security policies are a widely researched field.
Many systems tackle particular issues such as non-interference, cache side-channel attacks, or speculation exploits \cite{2017secverilogbl, 2019stt}.
As the attack model is ever-mutating and often undetected, these solutions can only ever solve part of the problem.
Some solutions propose generic monitors based on programmable hardware monitors \cite{2014pump} or languages for describing arbitrary policies \cite{2018garuda, 2014netkat}.
We build on the idea of verifiable policy design and empower \oldname to support Speculative and Out-of-Order Processors.
The benefits of Speculative and OOP microarchitectures are one of the most significant advancements in recent latency-mitigation research.
\sysname benefits from a refined information model and taint tracking, implemented without changes to the language itself.
These additions incur marginal processor overhead, and ring true to the spatial and power efficiency goals of the past.
We inject monitors into only one part of the processor and automate the comparator for a more streamlined design.
Finally, we will release this code, compiler, and language on GitHub.

\end{abstract}

%%%%%%%%%%%%%%%%%%%%%%%%%
%%% BEGIN             %%%
%%% TABLE OF CONTENTS %%%
%%%%%%%%%%%%%%%%%%%%%%%%%
%%% This will probably be removed at a later date,
%%% but it allows for easier navication now.
  \addcontentsline{toc}{section}{Contents}
  \tableofcontents
%%%%%%%%%%%%%%%%%%%%%%%%%
%%% END               %%%
%%% TABLE OF CONTENTS %%%
%%%%%%%%%%%%%%%%%%%%%%%%%

%%
%% The code below is generated by the tool at http://dl.acm.org/ccs.cfm.
%% Please copy and paste the code instead of the example below.
% %%
% \begin{CCSXML}
% <ccs2012>
%    <concept>
%        <concept_id>10010583.10010717.10010721</concept_id>
%        <concept_desc>Hardware~Functional verification</concept_desc>
%        <concept_significance>500</concept_significance>
%        </concept>
%    <concept>
%        <concept_id>10010583.10010682.10010689</concept_id>
%        <concept_desc>Hardware~Hardware description languages and compilation</concept_desc>
%        <concept_significance>300</concept_significance>
%        </concept>
%    <concept>
%        <concept_id>10010583.10010786.10010787.10010789</concept_id>
%        <concept_desc>Hardware~Emerging languages and compilers</concept_desc>
%        <concept_significance>500</concept_significance>
%        </concept>
%    <concept>
%        <concept_id>10010520.10010521.10010522.10010526</concept_id>
%        <concept_desc>Computer systems organization~Pipeline computing</concept_desc>
%        <concept_significance>100</concept_significance>
%        </concept>
%    <concept>
%        <concept_id>10002978.10003001.10003002</concept_id>
%        <concept_desc>Security and privacy~Tamper-proof and tamper-resistant designs</concept_desc>
%        <concept_significance>500</concept_significance>
%        </concept>
%    <concept>
%        <concept_id>10002978.10003001.10003599.10011621</concept_id>
%        <concept_desc>Security and privacy~Hardware-based security protocols</concept_desc>
%        <concept_significance>500</concept_significance>
%        </concept>
%  </ccs2012>
% \end{CCSXML}

% \ccsdesc[500]{Hardware~Functional verification}
% \ccsdesc[300]{Hardware~Hardware description languages and compilation}
% \ccsdesc[500]{Hardware~Emerging languages and compilers}
% \ccsdesc[100]{Computer systems organization~Pipeline computing}
% \ccsdesc[500]{Security and privacy~Tamper-proof and tamper-resistant designs}
% \ccsdesc[500]{Security and privacy~Hardware-based security protocols}

%%
%% Keywords
\keywords{hardware security, software verification, memory obfuscation}

%%
%% This command processes the author and affiliation and title
%% information and builds the first part of the formatted document.
\maketitle
%%%%%%%%%%%%%%%%%%%%%%%
%%% BEGIN SECTION 1 %%%
%%% INTRODUCTION    %%%
%%%%%%%%%%%%%%%%%%%%%%%
  \section{Introduction}\label{sec:intro}
    \sysname Intro
    \cite{2011gem5sim}
    \cite{2019stt}
    The rest of this thesis loosely follows a standard academic journal submission's organization.
    Section~\ref{sec:priorwork} summarizes a prior attack models and their proposed solutions.
    The \sysname language is described in detail in Section~\ref{sec:spec} and is demonstrated in Section~\ref{sec:proofs}.
    Section~\ref{sec:compile} describes the intermediate steps towards the compilation of \sysname into Verilog.
    Research done since this paper's release are shown in Section~\ref{sec:more}.
    Sections~\ref{sec:conclude} and~\ref{sec:coauthor} conclude and cite the work done by my colleagues and advisors.
%%%%%%%%%%%%%%%%%%%%%
%%% END SECTION 1 %%%
%%% INTRODUCTION  %%%
%%%%%%%%%%%%%%%%%%%%%

%%%%%%%%%%%%%%%%%%%%%%%
%%% BEGIN SECTION 2 %%%
%%% PRIOR WORK      %%%
%%%%%%%%%%%%%%%%%%%%%%%
  \section{Prior Work}\label{sec:priorwork}
    \begin{table}
      \centering
      \begin{tabular}{|c|l|l|} 
        \hline
          & Rumtime Monitors & Compiler Enforcements\\
        \hline
          Advantages
          & - Often programmable 
            & - Low Overhead \\
          & - Can trigger exceptions or
            & - Possible violations detected \\
          &\quad OS traps &\quad before synthesis \\ 
          & - Modular, not application  
            & - Violations during runtime \\
          &\quad specific &\quad simply fail \\
          \hline
          Disadvantages
          & - Exceptions and traps are slow 
            & - Static after synthesis \\
          & - Larger overhead 
            & - Violation-response is static \\
          & &\quad and defined at compile time \\
        \hline
      \end{tabular}
      \caption{Comparison of Runtime Monitors and Compile-Time Enforcement.}
      \label{table:prior:run-vs-comp}
    \end{table}

%%%%%%%%%%%%%%%%%%%%%
%%% END SECTION 2 %%%
%%% PRIOR WORK    %%%
%%%%%%%%%%%%%%%%%%%%%

  
%%%%%%%%%%%%%%%%%%%%%%
%%% BEGIN SECTON X %%%
%%% SYSTEM SPECS   %%%
%%%%%%%%%%%%%%%%%%%%%%
  \section {Syntax}
  %%%%%%%%%%%%%%%%%%%%%%%%%
  %%% BEGIN SECTION X.1 %%%
  %%% SYSTEM SYNTAX     %%%
  %%%%%%%%%%%%%%%%%%%%%%%%%
    \subsection{Syntax}\label{sec:spec:synt}
    %%%%%%%%%%%%%%%%%%%%%%%%%%%
    %%% BEGIN SECTION X.1.1 %%%
    %%% SYSTEM DEFINITIONS  %%%
    %%%%%%%%%%%%%%%%%%%%%%%%%%%
      \subsubsection{Definitions}\label{sec:spec:synt:defn}
        % Definitions table
        \begin{figure}
          \centering
          \begin{tabular}{l l c l}
            EXE Input Fields & $f_{Ei}$  & $::=$ & $f_{Ei_{1}} \mid \dots \mid f_{Ei_{k}}$\\
            State Reg Fields & $f_{SR}$  & $::=$ & $f_{SR_{1}} \mid \dots \mid f_{SR_{k}}$\\
            MEM Input Fields & $f_{Mi}$  & $::=$ & $f_{Mi_{1}} \mid \dots \mid f_{Mi_{k}}$\\
            Fields           & $f$       & $::=$ & $f_{Ei} \mid f_{SR} \mid f_{Mi} $ \\
            Obfuscated Field & $\obf{f}$ & $::=$ & $f$\\
            Inputs           & $i$       & $::=$ & $\{f_{i_{1}} = v_{i_{1}} ,\ \dots\ ,\ f_{i_{k}} = v_{i_{k}}\}$\\
            Outputs          & $o$       & $::=$ & $\{f_{o_{1}} = v_{o_{1}} ,\ \dots\ ,\ f_{o_{k}} = v_{o_{k}}\}$\\
            Obfuscation Fxn  & $\Phi$    & $::=$ & $f \rightarrow \obf{f} \mid \obf{f} \rightarrow f$
            % TODO: how do we define state in and state out? -- Who cares?  that's not part of the language
            % TODO: Mention that Inputs and Outputs can generically refer to either EXE or MEM.
            %       Indeed, the Outputs refers to both.
          \end{tabular}
          \caption{Definitions in \sysname.}
          \label{fig:spec:synt:defn}
        \end{figure}

    %%%%%%%%%%%%%%%%%%%%%%%%%%%%
    %%% BEGIN SECTION X.1.2  %%%
    %%% SYNTAX OF PREDICATES %%%
    %%%%%%%%%%%%%%%%%%%%%%%%%%%%
      \subsubsection{Syntax of Predicates}\label{sec:spec:synt:pred}
        % Predicates table
        \begin{figure}
          \centering
          \begin{tabular}{l c r l l}
            Predicates  & $a,b$ & $::=$  & $0$          & \textit{False}    \\
                        &       & $\mid$ & $1$          & \textit{Identity} \\
                        &       & $\mid$ & $f = n$      & \textit{Test}     \\
                        &       & $\mid$ & $a + b$      & \textit{Sum}      \\
                        &       & $\mid$ & $a \cdot b$  & \textit{Product}  \\
                        &       & $\mid$ & $\neg\ a$    & \textit{Negation}
          \end{tabular}
          \caption{Predicates in \sysname act as a boolean algebra.}
          \label{fig:spec:synt:pred}
        \end{figure}

    %%%%%%%%%%%%%%%%%%%%%%%%%%%
    %%% BEGIN SECTION X.1.3 %%%
    %%% SYNTAX OF POLICIES  %%%
    %%%%%%%%%%%%%%%%%%%%%%%%%%%
      \subsubsection{Syntax of Policies}\label{sec:spec:synt:pol}
        % Policies table
        \begin{figure}
          \centering
          \begin{tabular}{l c r l l}
            Policies  & $p,q$ & $::=$  & $\mathbf{test}(a)$ & \textit{Test}      \\
                      &       & $\mid$ & $(\Phi_{Encrypt}, 
                                           \Phi_{Decrypt})$ & \textit{Obfuscation} \\
                      &       & $\mid$ & $inj_{SR}$         & \textit{Injection to State Register} \\
                      % alternatively: Injection to State Register (inj_{i})
                      %   should I make this Rs?  That could be confused with the rs in MIPS, 
                      %   ... but we already used rs in GARUDA without problem
                      &       & $\mid$ & $inj_{Mi}$         & \textit{Injection to Memory Unit} \\
                      % alternatively: Injection from State Register (inj_{o})
                      &       & $\mid$ & $f \leftarrow n$   & \textit{Update}   \\
                      &       & $\mid$ & $p + q$            & \textit{Choice}   \\
                      &       & $\mid$ & $p \cdot q$        & \textit{Sequential Concatenation} \\
          \end{tabular}
          \caption{Syntax of Policies in \sysname}
          \label{fig:spec:synt:pol}
        \end{figure}



  %%%%%%%%%%%%%%%%%%%%%%%%%
  %%% BEGIN SECTION X.1 %%%
  %%% SYSTEM SEMANTICS  %%%
  %%%%%%%%%%%%%%%%%%%%%%%%%
    \subsection{Semantics}\label{sec:spec:sem}
    %%%%%%%%%%%%%%%%%%%%%%%%%%%%%
    %%% BEGIN SECTION X.1.1   %%%
    %%% SEMANTICS: PREDICATES %%%
    %%%%%%%%%%%%%%%%%%%%%%%%%%%%%
      \subsubsection{Semantics of Predicates}\label{sec:spec:sem:pred}

        % Semantics of Predicates
        \begin{figure}
          \centering
          \begin{align*}
            % Theory
            \interp{ \cdot }\ 
              :\ \ &
              \mathbf{Stream}(E)\times \mathbf{Stream}(M) \rightarrow \\
              & P(\mathbf{Stream}(E))\times P(\mathbf{Stream}(M)) 
              \\
            % Falsity
            \interp{ 0 }(-, -)
              \triangleq\ &
              (\emptyset , \emptyset)
              \\ % or null-stream?
            % Identity
            \interp{ 1 }(es, ms)
              \triangleq\ &
              (\{es\},\{ms\})
              \\
            % Test
            \interp{ f=n }(es, ms)
              \triangleq\ &
              (\mathsf{filter}\ (f=n)\ \{es\},\
               \mathsf{filter}\ (f=n)\ \{ms\}) 
              \\
            % Sum
            \interp{ a + b }(es, ms)
              \triangleq\ &
              \interp { a }(es, ms)\cup
              \interp { b }(es, ms) \\
              &\mathsf{where}\ (S_e^1, S_m^1)\cup (S_e^2, S_m^2)\triangleq
                (S_e^1\cup S_e^2, S_m^1\cup S_m^2)\\
            % Product
            \interp { a \cdot b }(es, ms)
              \triangleq\ &
              \interp { a }(es, ms)\cap
              \interp { b }(es, ms) \\
              &\mathsf{where}\ (S_e^1, S_m^1)\cap (S_e^2, S_m^2)\triangleq
                (S_e^1\cap S_e^2, S_m^1\cap S_m^2)\\
            % Negation
            \interp { \neg a }(es, ms)
              \triangleq\ &
              \mathsf{let}\ (S_e, S_m) = \interp {a}(es, ms) \\
              &\mathsf{in}\ (\{es\} - S_e, \{ms\} - S_m)
              \\
          \end{align*}
          \caption{The semantics of predicates in \sysname.  While predicates remain unchanged from \oldname, the theory changes slightly}
          \label{fig:garuda:sem:pol}
        \end{figure}

    %%%%%%%%%%%%%%%%%%%%%%%%%%%
    %%% BEGIN SECTION X.1.2 %%%
    %%% SEMANTICS: POLICIES %%%
    %%%%%%%%%%%%%%%%%%%%%%%%%%%
      \subsubsection{Semantics of Policies}\label{sec:spec:sem:pol}
        \begin{figure}
          \centering
          \begin{align*} 
            % Theory
            \interp { \cdot }\ 
              :\ \ &
              \mathbf{Stream}(E)\times \mathbf{Stream}(M) \rightarrow \\
              & P(\mathbf{Stream}(E))\times P(\mathbf{Stream}(M)) 
              \\
            % Obfuscation
            \interp {(\Phi_{Encrypt}, \Phi_{Decrypt})}(es, ms)\
              \triangleq\
              & \mathsf{let}\ (e \rightarrow \obf{e} = \Phi_{Encrypt}),
                            \ (\obf{m} \rightarrow m = \Phi_{Decrypt})\\
              % & \mathsf{let}\ e \rightarrow \obf{e} = \Phi_{Encrypt}\\
              % & \mathsf{and}\ \obf{m} \rightarrow m = \Phi_{Decrypt}\\
              & \mathsf{in}\
              (\{\obf{e}\}, \{m\})
              \\
            % Injection Action
            \interp { inj_{SR}(e) }(es, ms)
              \triangleq\ &
              (\{e : es\}, \{ms\}) 
              % or ::?  ;i sequential
              \\
            % Injection Result
            \interp { inj_{Mi}(m) }(es, ms)
              \triangleq\ &
              (\{es\},\{m : ms\})
              \\
            % Modification
            \interp { f \leftarrow n }(es, ms)
              \triangleq\ &
              (\mathsf{map}\ (f\leftarrow n)\ \{e\},\
               \mathsf{map}\ (f\leftarrow n)\ \{m\})
              \\ %  could use editing?
            % Policy Union
            \interp { p + q }(es, ms)
              \triangleq\ &
              \interp { p }(e, m)\cup
              \interp { q }(e, m) \\
              &\mathsf{where}\ (S_e^1, S_m^1)\cup (S_e^2, S_m^2)\triangleq
                (S_e^1\cup S_e^2, S_m^1\cup S_m^2)\\
              % A union of the streams gives a nondeterministic machine.
            % Policy Concatnation
            \interp { p \cdot q }(es, ms)
              \triangleq\ &
              \mathsf{let}\ (S_e, S_m) = \interp{p}(e, m)\\
              &\mathsf{in}\ \bigcup \{\interp{q}(e',m')\ |\ e'\in S_e, m'\in S_m\}\\
              \\
            % Defn. of Filter
            \interp{\mathsf{filter}\ f}(S)
              \triangleq\ & \{l \in S\ |\ f(l) = \mathsf{true}\}\\
            % Defn. of Map
            \interp{\mathsf{map}\ g}(S)
              \triangleq\ &
              \{ g(l)\ |\ l\in S \} 
          \end{align*}
          \caption{The semantics of policies in \sysname.}
          \label{fig:spec:sem:pol}
        \end{figure}
%%
%% The acknowledgments section is defined using the "acks" environment
%% (and NOT an unnumbered section). This ensures the proper
%% identification of the section in the article metadata, and the
%% consistent spelling of the heading.
\begin{acks}
To ...
\end{acks}

%%
%% The next two lines define the bibliography style to be used, and
%% the bibliography file.
\bibliographystyle{ACM-Reference-Format}
\bibliography{Garuda2.0}

%%
%% If your work has an appendix, this is the place to put it.
\appendix

\section{Research Methods}

\subsection{Part One}

Appendix A1

\subsection{Part Two}

Appendix A2

\section{Online Resources}

Appendix B

\end{document}
\endinput
